\PlakatUeberschrift{Schriftgröße 11 pt}

Dies ist die Vorlage für das Plakat (DIN \PlakatFormat{}) der Technischen
Universität München (TUM). Sie entspricht dem Corporate Design der TUM und ist für das Betriebssystem Windows getestet und mit MiKTeX 2.9 kompatibel.

Bitte geben Sie Ihren individuellen Text an den vorgesehenen Stellen ein. Nutzerhinweise finden Sie außerdem in der Liesmich.txt.

\ifx\PlakatBeschreibungKopfzeileUndAbsender\TRUE
\PlakatUeberschrift{Kopfzeile und Absender}

Es stehen mehrere Varianten zur Auswahl. Diese finden Sie auf den nächsten Seiten als Beispiel oder unter \href{http://\UniversitaetWebseite/cd}{\UniversitaetWebseite/cd}, welche für die verschiedenen Nutzergruppen vorgesehen sind.

\fi

\PlakatUeberschrift{Text}

Strukturieren Sie Ihr Plakat möglichst klar und beschränken Sie sich auf die wichtigsten Informationen. Entscheiden Sie sich für einen ein- oder mehrspaltigen Aufbau, linksbündig oder im Blocksatz. Und wählen Sie bei wenig Text eine entsprechend größere Schrift.

\PlakatUeberschrift{Bilder}

\ifx\PlakatBeschreibungKurz\TRUE % BEGINN kurze/lange Beschreibung

Auch wenn Sie Bilder integrieren möchten, können Sie innerhalb des vorgegebenen Rahmens variieren:

\begin{itemize}
\item Das Bild ist an die Breite der Textspalte angepasst
\item Das Bild geht über alle Textspalten
\item Das Bild ist seitenfüllend innerhalb des vorgesehenen Textrahmens (mit weißem Rand)
\item Das Bild ist seitenfüllend bis an den Papierrand (randabfallend) an mindestens drei Seiten
\end{itemize}

Bilder sollten immer eine Bildunterschrift tragen, die ggf. Informationen zum Bild und zur Autorenschaft enthält.
Das Bild sollte immer genügend Abstand zum Text haben. Gleiches gilt für Grafiken.

\else

Auch wenn Sie Bilder integrieren möchten, können Sie innerhalb des vorgegebenen Rahmens variieren:

\begin{itemize}
\item Das Bild ist an die Breite der Textspalte angepasst
\item Das Bild geht über alle Textspalten
\item Das Bild ist seitenfüllend innerhalb des vorgesehenen Textrahmens (mit weißem Rand)
\item Das Bild ist seitenfüllend bis an den Papierrand (randabfallend) an mindestens drei Seiten
\end{itemize}

Bilder sollten immer eine Bildunterschrift tragen, die ggf. Informationen zum Bild und zur Autorenschaft enthält. 
Das Bild sollte immer genügend Abstand zum Text haben. Gleiches gilt für Grafiken.

\ifx\PlakatBeschreibungBeispielbild\TRUE
\PlakatBild[0cm \PlakatBeschreibungBeispielbildBeschnitt{} 0cm 0cm]{./Ressourcen/_Bilder/Sternenhimmel.jpg}{Bildunterschrift, Autor etc}
\fi

\ifx\PlakatBeschreibungDruck\TRUE
\PlakatUeberschrift{Druck}

Achten Sie darauf, dass das Dokument in Originalgröße gedruckt wird, also keine Anpassung an die Druckränder bei Ihren Druckereinstellungen aktiviert ist.
Wählen Sie bitte unbedingt weiße Papiere statt Naturpapier mit starker Braun- oder Graufärbung. Denn nur sie entsprechen dem Charakter des Corporate Designs der TUM. Denken Sie daran, dass auch Papierstärke (Dicke) und Haptik einen Einfluss auf das Druckergebnis haben.
\fi

\fi % ENDE kurze/lange Beschreibung

Weitere Infos zum Corporate Design der TUM finden Sie unter
\href{http://\UniversitaetWebseite/cd}{\UniversitaetWebseite/cd}

