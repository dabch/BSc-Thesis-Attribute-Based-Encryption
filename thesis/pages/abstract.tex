\chapter{\abstractname}

Attribute-Based Encryption (ABE) enforces flexible access control by allowing implicit specification of intended recipients.
Ciphertexts and users are described by attributes, and decryption is possible only if they match according to a given policy.
The flexibility of ABE and expressiveness of its policies make it very useful for many scenarios, e.g. encrypted communication with multiple addressees, especially if the group of recipients is not predetermined.

Most ABE schemes are constructed with bilinear pairings of elliptic curves.
These are computationally very expensive, and thus ABE has found limited use the IoT.

This thesis gives an overview of ABE and evaluates its applicability on ARM Cortex-M4 processors.
The runtime, RAM usage and size of the executable are evaluated on a SoC with a 64\,MHz Cortex-M4 CPU and 256\,KB RAM.
Two Key-Policy ABE schemes are implemented in a library for embedded systems; one of the schemes is a pairing-free scheme.
% This library is then evaluated on an ARM Cortex-M4-based SoC.
While the security of pairing-based schemes is better, the pairing-free scheme provides significantly better performance.
Encryption is much slower with the pairing-based scheme, but feasible with both schemes if a delay of a few seconds is acceptable and the number of attributes is not too large.
Decryption is only practical with the pairing-free scheme.