\chapter{\abstractname}

%TODO: Abstract

Attribute-Based Encryption is a type of encryption that enforces flexible access control policies on the encrypted data.
Ciphertexts and users are described by attributes, and decryption is possible only if they match according to the specified policy.
Because ABE is computationally expensive, its feasibility on constrained devices is questionable.

This thesis gives an overview of ABE and evaluates its applicability on ARM Cortex M4 processors.
Two KP-ABE schemes are implemented in a library for embedded systems; one of the schemes is a pairing-free scheme.
% This library is then evaluated on an ARM Cortex M4-based SoC.
The result is that the pairing-free scheme performs significantly better than the pairing-based scheme.
Encryption is feasible on the SoC with both schemes if the number of attributes is not too large and a runtime of a few seconds is acceptable.
Decryption is similarly feasible with the pairing-free scheme for all tested policy sizes.
With the pairing-based scheme, it is only feasible for small policies and fails entirely for large policies.



