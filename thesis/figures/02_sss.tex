\begin{figure}
    \centering
    \begin{tikzpicture}[scale=0.7]
        \begin{axis} [axis lines=center, xlabel=$x$, ylabel=$p(x)$, ymin=0]
            \addplot [domain=-0.1:5.05, smooth, thick, color=red] { \PolynomialSSS(x)};
            \fill (axis cs:0,\PolynomialSSS(0)) circle [radius=2pt] node [right] {(0,~s)};
            \fill[color=ForestGreen] (axis cs:1,\PolynomialSSS(1)) circle [radius=2pt] node [right] {(1,~10)};
            \fill[color=ForestGreen] (axis cs:2,\PolynomialSSS(2)) circle [radius=2pt] node [right] {(2,~6)};
            \fill[color=ForestGreen] (axis cs:3,\PolynomialSSS(3)) circle [radius=2pt] node [above] {(3,~2)};
            \fill[color=ForestGreen] (axis cs:4,\PolynomialSSS(4)) circle [radius=2pt] node [right] {(4,~4)};
            \fill[color=ForestGreen] (axis cs:5,\PolynomialSSS(5)) circle [radius=2pt] node [left] {(5,~18)};
          \end{axis}
    \end{tikzpicture}
    \caption[Plot of $(5,4)$-threshold secret sharing scheme]{
        Example for a $(5, 4)$-threshold scheme with $s=8$ and $p(x) = 8 + 7x - 6x^2 + x^3$.
        The five green-colored points are distributed as the secret shares.
        Because $p(x)$ has degree three, at least four shares are required to reconstruct $s$.
    }
    \label{fig:sss}
\end{figure}