\begin{figure}[h]\centering
    \begin{tikzpicture}
        \begin{axis}[
            title={RAM use during decryption},
            width=0.7\textwidth,
            height=0.4\textheight,
            xlabel={depth of the \gls{access-tree}},
            ylabel={kilobytes},
            xmin=0.75, xmax=4.25,
            ymin=0, ymax=360000,
            xtick=data,
            scaled y ticks=base 10:-3,
            ytick scale label code/.code={}, % removes the '\cdot 10^3' label 
            legend pos=north west,
            grid=major,
            minor y tick num=1,
            grid style=dashed,
            legend style={nodes={scale=1, transform shape}}
        ]
        \addlegendentry{GPSW};
        \addplot [color=blue, mark=square] table [x=layers,y=gpsw,col sep=comma] {data/ram.csv};
        \addlegendentry{YCT};
        \addplot [color=orange, mark=triangle] table [x=layers,y=yct,col sep=comma] {data/ram.csv};
        % \addlegendentry{SoC RAM size};
        % \addplot [color=black, mark=none] {262144};
        \end{axis}
    \end{tikzpicture}
    \caption[RAM use of ABE decryption]{RAM use of \acrshort{abe} decryption on the laptop. The \glspl{access-tree} used for key geneneration are \glspl{perfect-binary-tree} of the given depth. A depth of one means that the tree consists of one root with two children.}
    \label{fig:abe-performance-diagrams-ram}
\end{figure}