\chapter{Evaluation}

\section{Performance of \texttt{rabe-bn}}
\begin{table}\centering
    \begin{tabular}{|c|r|r|}\hline%
        & SoC [ms] & Laptop [ms]\\\hline\hline
        \csvreader[late after line=\\]%
        {data/bn-smpl.csv}{op=\op,soc=\soc,laptop=\laptop}%
        {\op&\soc&\laptop}%
        \hline
        \csvreader[late after line=\\]%
        {data/bn-groupop.csv}{op=\op,soc=\soc,laptop=\laptop}%
        {\op&\soc&\laptop}%
        \hline
        \csvreader[late after line=\\]%
        {data/bn-groupexp.csv}{op=\op,soc=\soc,laptop=\laptop}%
        {\op&\soc&\laptop}%
        \hline
        \csvreader[late after line=\\]%
        {data/bn-pairing.csv}{op=\op,soc=\soc,laptop=\laptop}%
        {\op&\soc&\laptop}%
        \hline
    \end{tabular}  
    \caption{Execution times for various operations on the SoC and the laptop\label{tbl:rabe-performance}}
\end{table}

See Table~\ref{tbl:rabe-performance} for performance measurements of random element sampling, group operations, group-scalar exponentiation and the pairing operation.
The times have been measured using randomly sampled elements and averaged over 100 calls each.

It is 

% \csvreader[tabular=|r|r|r|,
%     table head=\hline & Soc [ms] & Laptop [ms]\\\hline\hline,
%     late after line=\\\hline]
%     {data/smpl-times.csv}{description=\description,soc=\soc,laptop=\laptop}
%     {\thecsvrow & \description & \soc & \laptop}

% \begin{tabular}{|r|r|r|}\hline%
%     & SoC [ms] & Laptop [ms]\\\hline
%     \csvreader[late after line=\\]%
%         {data/smpl-times.csv}{description=\description,soc=\soc,laptop=\laptop}%
%         {\description & \soc & \laptop}%
% \end{tabular}\


\dots 

RAM size seems to be a limiting factor when computing pairings on embedded devices.
During development, the ported \texttt{rabe-bn} library was also tested on the nRF52832 SoC, which has 64KB of RAM (vs. 256KB in the nRF52840). 
On this chip, a pairing could be computed successfully only if the library was build \emph{without debug symbols}.
With debug symbols, there was not enough RAM available and the pairing computation failed.
This suggests that the memory use during pairing computation is close to 64KB, which would still be a quarter of the RAM on the nRF52840 SoC.
While this memory is not consumed permanently, it still needs to be available when the pairing function is called.

\section{Experimental results}

\chapter{Discussion}

\chapter{Conclusion}