\chapter{Related Work}

% \begin{itemize}
%     \item Small / Large attribute universe
%     \item efficiency: pairings, exponentiations
%     \item size of: ciphertext, keys, key updates (revocation stuff)
%     \item expressiveness / expression of access policies
%     \item Revocation
% \end{itemize}

% TODO
% a) intro attribute-based stuff (Sahai & Waters etc.)
%    - basics (fuzzy IBE, Goyal, Bethencourt) 
%    - multi-authority
%    - revocation, non-monotonic access structures
%    - hierachical capabilities
%    - ciphertext, private key, public parameters sizes
% b) schemes for IoT (using conventional crypto, pairing-free, server-aided etc.)
% c) implementation works (IoT, BAN, pairings on microcontrollers)

\acrlong{abe} was introduced by Sahai and Waters in 2005~\cite{sahai_fuzzy_2005}.
They proposed a new type of \gls{ibe} where identities are a set of attributes.
Their \emph{fuzzy} \gls{ibe} scheme allows a user to decrypt a ciphertext even if their identity doesn't exactly match the identity specified at the time of encryption~\cite{sahai_fuzzy_2005}.
Instead, an overlap larger than some threshold value between the attributes in the ciphertext's identity with the attributes of the key's identity is sufficient~\cite{sahai_fuzzy_2005}.
This property is realized by means of a $(k, n)$-threshold secret sharing scheme.

Sahai and Water's construction can already be seen as an \acrshort{abes} with very limited expressiveness, i.e. it only works with ''k-out-of-n'' access structures~\cite{goyal_attribute-based_2006}.

In 2006, Goyal, Pandey, Sahai and Waters~\cite{goyal_attribute-based_2006} extended this into the first expressive \acrshort{kp-abe} scheme using the \gls{access-tree} construction described in Chapter~\ref{chapter:background}.
Their main construction uses \glspl{access-tree} and a small attribute universe, but they also give constructions with a large attribute universe and for \gls{lsss} access structures, respectively.

The first expressive \acrshort{cp-abe} scheme was proposed by Bethencourt, Sahai and Waters in~\cite{bethencourt_ciphertext-policy_2007}.
It is also a large-universe construction and uses \glspl{access-tree}.
Waters \cite{waters_ciphertext-policy_2011} also gives the first \acrshort{cp-abe} schemes with a security proof in the \gls{standard-model}, not only in the \gls{ggm} (the distinction is not relevant for this thesis).

Both the schemes in~\cite{goyal_attribute-based_2006} and in~\cite{bethencourt_ciphertext-policy_2007} only support monotonic \glspl{access-structure}.

In \cite{goyal_attribute-based_2006}, an inefficient realization of general (non-monotonic) \glspl{access-structure} is proposed, which is to simply represent the absence of an attribute as a separate attribute.
This is inefficient because it doubles the total number of attributes in the system~\cite{goyal_attribute-based_2006}. 
Non-monotonic access structures over a universe of $n$ attributes are represented by monotonic access structures over a universe of $2n$ attributes.
It also requires every ciphertext to be associated with $n$ attributes (i.e. either with their positive or negated of a corresponding attribute).
Note that the size of ciphertexts or keys is linear in the number of attributes in all expressive \acrshortpl{abes}. % TODO make sure that this is true

The first efficient construction for non-monotonic \glspl{access-structure} was given in~\cite{ostrovsky_attribute-based_2007}. 
However, this construction leads to private keys of relatively large size.
More specifically, the size is $\mathcal{O}(t \log(n))$, where $t$ is the number of leaf nodes in the key's \gls{access-tree} and $n$ a system-wide bound on the number of attributes a ciphertext may have~\cite{lewko_revocation_2008}. % TODO be more explicit about the size

The scheme in \cite{lewko_revocation_2008} provides efficient direct revocation.
It also relates direct revocation to non-monotone \glspl{access-structure}.
\cite{lewko_revocation_2008} provides an efficient construction for non-monotone access structures with keys of size $\mathcal{O}(t)$~\cite{lewko_revocation_2008}, where $t$ is as above.

The difference between direct and indirect revocation is introduced in \cite{attrapadung_attribute-based_2009}, and a \emph{Hybrid Revocable} \acrshort{abes} is given.
It allows the encryptor to choose the revocation mode separately for every message~\cite{attrapadung_attribute-based_2009}.

All of these schemes are built using a bilinear pairing as introduced in Section~\ref{sec:bilinear-pairings}.
% TODO add some stuff on server-aided IoT ABE and conventional constructions for ABE
A pairing-free \acrshort{kp-abe} scheme was proposed by Yao, Chen and Tian \cite{yao_lightweight_2015} in 2015.
Their scheme only uses a single group and no bilinear pairing.
Instead of encrypting a group element that encodes a message, their scheme yields a random group element which is then used as a key for a symmetric encryption algorithm~\cite{yao_lightweight_2015}.

In \cite{tan_enhancement_2019} a cryptanalysis of the scheme in \cite{yao_lightweight_2015} is performed.
It is shown that the scheme is not secure, but the authors propose an effective fix and prove its security.
They also extend the scheme to allow for key delegation (i.e. a hierarchical \acrshort{kp-abe} scheme)~\cite{tan_enhancement_2019}.

\cite{sowjanya_efficient_2020} presents a pairing-free \acrshort{abes} with indirect revocation.
It is an adaptation of the schemes in \cite{yao_lightweight_2015,tan_enhancement_2019}, see also \cite{herranz_attacking_2020}.

All three of these schemes were attacked by Herranz in \cite{herranz_attacking_2020} (one attack for all three schemes is given, as they are very similar).
\cite{herranz_attacking_2020} argues that it is not possible to build secure \acrshortpl{abes} in the (non-bilinear) discrete-logarithm setting (i.e. on elliptic curves without bilinear pairings).
This remains to be assessed.