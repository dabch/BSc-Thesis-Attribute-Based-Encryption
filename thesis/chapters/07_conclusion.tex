\chapter{Summary and Conclusion}

The aim of this thesis was to present \acrshort{abe} and evaluate its practical feasibility on constrained \acrshort{iot} nodes.
More specifically, I implemented an \acrfull{abe} library with two schemes and evaluated it on an ARM Cortex M4 SoC.
To this end, the underlying elliptic curve and pairing library \texttt{rabe-bn} had to be re-written to run on embedded systems.
For the evaluation of the library, all four \acrshort{abe} algorithms were tested on the embedded SoC and on the laptop.\\

Computation in the relevant elliptic curve groups and their pairings is possible on the SoC with our library, but comes at a rather high price.
Comparison with other implementations shows that there is room for improvement here.
The true bottleneck for pairing-based implementation seems to be the memory size:
Pairing computation failed on a different SoC with less RAM and the pairing-based GPSW scheme was limited to smaller policies.

In the use case presented in the introduction, the constrained node only needs to encrypt data with \acrshort{abe}.
For a small or medium number of attributes, this definitely is feasible.
The encryption runtime is in the order of a few seconds and only increases linearly as more attributes are added.
For use cases where such a delay is not practical, re-use of the encapulated symmetric key might mitigate the issue.

With decryption, the case is different: 
No problems were encountered with the pairing-free YCT scheme.
Indeed, decryption was faster than encryption for the randomly chosen policies in the test set.
With GPSW, however, encryption failed for a larger number of attributes because of insufficient RAM.
Also, runtimes were longer than for encryption and increased much more significantly with growing policy size.
While long runtimes might sometimes be acceptable, 

If decryption with large policies is necessary on the SoC, it is therefore advisable to choose YCT over GPSW.
The security of pairing-free \acrshortpl{abes}, however, remains questioned (see~\cite{herranz_attacking_2020}).\\

In short, \acrfull{abe} seems feasible on the considered hardware, especially if only encryption is needed and the number of attributes isn't too large.
Still, \acrshort{abe} comes at a high price in runtimes and memory consumption.
Especially decryption with the pairing-based scheme is infeasible if large policies are used.