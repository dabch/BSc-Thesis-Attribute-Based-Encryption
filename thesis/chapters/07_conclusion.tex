\chapter{Summary and Conclusion}

In this thesis, I presented \acrshort{abe} and evaluated its practical feasibility on constrained \acrshort{iot} nodes.
More specifically, I implemented an \acrfull{abe} library with two schemes and evaluated it on an ARM Cortex M4 SoC.
To this end, I had to modify the underlying elliptic curve and pairing library \texttt{rabe-bn} to run on embedded systems.
For the evaluation of the library, all four \acrshort{abe} algorithms (Encrypt, Decrypt, Setup and KeyGen) were tested on the embedded SoC.

Computing the underlying primitives for \acrshort{abe} on the SoC is feasible, but comes at a rather high price.
Comparison with other pairing implementations shows that there is some room for improvement here, but the the operations remain expensive.
In addition to computation time, memory size is a major bottleneck for pairing-based implementations.
% The GPSW scheme is limited to small and medium policies for key generation and decryption. 
% With less memory, even computation of a single pairing fails because of insufficient RAM.

In the use case presented in the introduction, the constrained node only needs to encrypt data with \acrshort{abe}.
For a small or medium number of attributes, this definitely is feasible:
RAM size was sufficient even with GPSW and large policies.
The runtime is in the order of a few seconds and only increases linearly as more attributes are added.
For use cases where such a delay is not practical, re-use of the encapsulated symmetric key might mitigate the issue.

With decryption, the case is different: 
No problems were encountered with the pairing-free YCT scheme.
Indeed, decryption was faster than encryption for the randomly chosen policies in the test set.
With GPSW, however, encryption failed for a larger number of attributes because of insufficient RAM.
Also, runtimes were longer than for encryption and increased much more significantly with growing policy size.
While long runtimes might sometimes be acceptable, the RAM limitation poses a hard limit.

If decryption with large policies is necessary on the SoC, it is therefore advisable to choose YCT over GPSW.
The security of pairing-free \acrshortpl{abes}, however, remains questionable, see~\cite{herranz_attacking_2020}.
Further research in this area is needed.\\

In short, \acrfull{abe} seems feasible on the considered hardware, especially if only encryption is needed and the number of attributes isn't too large.
Still, \acrshort{abe} comes at a high price in runtimes and memory consumption.
Especially decryption with the pairing-based scheme is infeasible if large policies are used.

% TODO something along the lines of "ABE takes up significant portion of the available resourcces"