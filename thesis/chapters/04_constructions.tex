\chapter{Evaluated ABE schemes}\label{chapter:constructions} 

This chapter will describe the two \acrshort{abe} schemes in detail that were implemented in this thesis.
In addition to a detailed description of the schemes, any modifications from the original definitions are made clear.

Both implemented schemes are \acrshort{kp-abe}.
This choice was made because \acrshort{kp-abe} is better suited to our use case from Figure~\ref{fig:system-architecture} (see Section~\ref{fig:cp-kp-abe}).
Also, encryption tends to be more efficient than with \acrshort{cp-abe}.

The GPSW scheme was chosen because it was the first expressive \acrshort{kp-abe} scheme. It is also considered a rather efficient scheme, compared to others that use bilinear pairings~\cite{girgenti_feasibility_2019}.
The YCT scheme was chosen for its unique approach without bilinear pairings. Because pairings are computationally expensive, this promises better performance.

\section{Goyal, Pandey, Sahai and Waters, 2006}
This scheme was the first \acrshort{abes} with expressive \glspl{access-policy}. Policies are associated with the key (\acrshort{kp-abe}).
It was described by Goyal, Pandey, Sahai and Waters \cite{goyal_attribute-based_2006} in 2006. This scheme will be referred to as GPSW.

Goyal~et.~al. extend the earlier work from Sahai and Waters~\cite{sahai_fuzzy_2005} to allow arbitrary access structures expressed by \glspl{access-tree}, not just ``k-out-of-n'' attributes.
They are the first to use Shamir's Secret Sharing hierarchically in the \gls{access-tree} as described in Section~\ref{sec:lss-in-access-trees}. 

The GPSW scheme encrypts a message represented by a point of the bilinear pairing's target group $\mathbb{G}_T$.
It is a \gls{small-universe} construction.

The scheme is defined here exactly as implemented; it differs from the original construction in the use of an asymmetric pairing ($e: \mathbb{G}_1 \times \mathbb{G}_2 \rightarrow \mathbb{G}_T$) instead of a symmetric pairing ($e: \mathbb{G}_1 \times \mathbb{G}_1 \rightarrow \mathbb{G}_T$).

In the GPSW construction, the pairing is evaluated when the decryption algorithm encounters a leaf node (see below).
There, the curve point on one side comes from the ciphertext, and the point on the other side from the key.
Originally, a symmetric pairing is used, so the pairing inputs can be swapped freely.
As we want to improve the speed of the encryption, we swap the two groups to use $\mathbb{G}_1$ for the group elements associated with ciphertexts.
Elements of $\mathbb{G}_1$ are shorter than elements of $\mathbb{G}_2$ and thus faster to calculate with.

% To speed up encryption and decryption, the plaintext is not encrypted with the GPSW \acrshort{abes} directly.
% Instead, a random group element is chosen and encrypted under GPSW (i.e. a $k \in \mathbb{G}_T$).
% This element is hashed to obtain an symmetric key, which is then used to encrypt the plaintext with AES-GCM (an \acrshort{aead} mode of operation).
% The ciphertext now consists of the GPSW-encrypted group element plus the AES-GCM ciphertext.

~\\

Let $\mathbb{G}_1$ and $\mathbb{G}_2$ be bilinear groups of prime order $q$. Let $P$ be a generator of $\mathbb{G}_1$ and $Q$ be a generator of $\mathbb{G}_2$. Let $e: \mathbb{G}_1 \times \mathbb{G}_2 \rightarrow \mathbb{G}_T$ be a bilinear map.
Note that $\mathbb{G}_1$ and $\mathbb{G}_2$ are written additively, but $\mathbb{G}_T$ is written using multiplicative notation.\\

\emph{Setup}~\cite{goyal_attribute-based_2006}.
The attribute universe is defined as $\text{U} = \{1, 2, \dots, n\}$ and is fixed.
For every attribute $i \in U$, choose uniformly at random a secret number $t_i \in \mathbb{Z}_q$.
Then the public key of attribute $i$ is $T_i = t_1 \cdot P$.
Also, choose uniformly at random the master private key $y \in \mathbb{Z}_p$, from which the master public key $Y = e(P, Q)^y$ is derived.

Publish $Params=(Y, T_1, \dots, T_n)$ as the public parameters, privately save $MK = (y, t_1, \dots, t_n)$ as the master key.
\\

\emph{Encrypt(M, $\omega$, Params)}~\cite{goyal_attribute-based_2006}.
Input: Message $M \in \mathbb{G}_T$, set of \glspl{attribute} $\omega$ and public parameters $Params$. 

Choose $s \in \mathbb{Z}_q$ at random and compute $E' = M \cdot Y^s$.
For each attribute $i \in \omega$ compute $E_i = s \cdot T_i$.

Return the ciphertext as $E = (\omega, E', \{E_i | i \in \omega\})$
\\

\emph{KeyGen($\Gamma$, MK)}~\cite{goyal_attribute-based_2006}.
Input: \gls{access-tree} $\Gamma$ and master key $MK$.

For each node $u$ in the \gls{access-tree} $\Gamma$, recursively define polynomials $q_u(x)$ with degree $(d_u - 1)$, starting from the root.

For the root $r$, set $q_r(0) = s$ and randomly choose $d_r -1$ other points to determine the polynomial $q_r(x)$.
Then, for any other node $u$, including leaf nodes, set $q_u(0) = q_{\text{parent}(x)}(\text{index}(x))$ and choose $d_u -1$ other points at random to define the polynomial. 
For all leaf nodes $u$, create a secret share $D_u = q_x(0) \cdot t_i^{-1} \cdot Q$ where $i = \text{att}(x)$.

The set of these secret shares is the decryption key $D = \{D_u | u \text{ leaf node of } \Gamma\}$.
\\

\emph{Decrypt(E, D)}~\cite{goyal_attribute-based_2006}.
Input: Ciphertext $E$ and decryption key $D$.

First, define a recursive procedure $\text{DecryptNode}(E, D, u)$ which takes as inputs a ciphertext $E = (\omega, E', \{E_i | i \in \omega\})$, the decryption key $D$ and a node $x$ of the \gls{access-tree} associated with the decryption key.
It outputs either en element of $\mathbb{G}_T$ or $\perp$.

If $u$ is a leaf node, then $i = \text{att}(x)$ and 
\begin{equation}
    \text{DecryptNode}(E, D, u) = \begin{cases}
        e(E_i, D_u) = e(s \cdot t_i \cdot P, q_u(0) \cdot t_i^{-1} \cdot Q) = e(P, Q)^{s\cdot q_u(0)} & i \in \omega\\
        \perp & i \notin \omega
    \end{cases}\\
\end{equation}

If $u$ is not a leaf node, instead call $\text{DecryptNode}(E, D, v)$ for all child nodes $v$ of $u$ and store the result in $F_v$.
Let $S_u$ be an arbitrary $d_u$-sized subset of child nodes $v$ with $F_v \neq \perp$. If no such set exists, the node was not satisfied. In this case return $\perp$.
Then compute with $i = \text{index}(z)$ and $S'_u = \{\text{index}(z) | z \in S_u\}$.
\begin{equation}
    \begin{split}
        F_u &= \prod_{z \in S_u} F_z^{\Delta_{i,S'_u}(0)}\\
        &= \prod_{z \in S_u} (e(P,Q)^{s\cdot q_z(0)})^{\Delta_{i,S'_u}(0)}\\
        &= \prod_{z \in S_u} (e(P,Q)^{s\cdot q_{\text{parent}(z)}(\text{index}(z))})^{\Delta_{i,S'_u}(0)}\\
        &= \prod_{z \in S_u} e(P,Q)^{s\cdot q_u(i) \cdot \Delta_{i,S'_u}(0)}\\
        &\stackrel{(*)}{=} e(P,Q)^{s \cdot q_u(0)}
    \end{split}
\end{equation}

The equality $(*)$ holds because, in the exponent, the product becomes a sum: $\sum_{i\in S'_u} s \cdot q_u(i) \cdot \Delta_{i,S'_u}(0)$ is exactly the lagrange interpolation of $s \cdot q_u(0)$.

Let the root of the \gls{access-tree} be $r$, then the decryption algorithm simply calls $\text{DecryptNode}(E, D, r) = e(P,Q)^{s \cdot y} = Y^s$, if the ciphertexts's attributes satisfy the \gls{access-tree}.
If they don't, then $\text{DecryptNode}(E, D, r) = \perp$.

To retrieve the message from $E' = M \cdot Y^s$, simply calculate and return $M' = E' \cdot (Y^s)^{-1}$.

% Of course, it is rather difficult (and slow) to encode the full plaintext as a group element of $\mathbb{G}_T$.
% Therefore, it is advisable to simply generate a random $K \in \mathbb{G}_T$ and encrypt the plaintext using a secure symmetric cipher with key $k = \text{KDF}(K)$, where $\text{KDF}$ is a \gls{kdf}.
% Then encrypt the point $K$ using the GPSW scheme and attach its ciphertext to the symmetric ciphertext.
% Correct decryption of $K \in \mathbb{G}_T$ then allows a receiver to decrypt the actual payload.

\section{Yao, Chen and Tian 2015}\label{sec:yct}

This scheme was described by Yao, Chen and Tian \cite{yao_lightweight_2015} in 2015.
In 2019, Tan, Yeow and Hwang \cite{tan_enhancement_2019} proposed an enhancement, fixing a flaw in the scheme and extending it to be a hierarchical KP-ABE scheme.

Yao, Chen and Tian's ABE scheme (hereafter written just YCT) is a KP-ABE scheme that does not use any bilinear pairing operations.
Instead, the only operation performed on Elliptic Curves are point-scalar multiplication~\cite{yao_lightweight_2015}.
% This makes it especially useful for our resource-constrained context, as bilinear pairings are significantly more costly in terms of computation and memory.

As opposed to other ABE schemes based on pairings, YCT uses a hybrid approach similar to the Elliptic Curve Integrated Encryption Standard (ECIES):
The actual encryption of the plaintext is done by a symmetric cipher, for which the key is derived from a curve point determined by the YCT scheme~\cite{yao_lightweight_2015}.
If a key's \gls{access-structure} is satisfied by a certain ciphertext, this curve point and thus the symmetric encryption key can be reconstructed, allowing for decryption.

The original description of this scheme uses the x- and y-coordinates as keys for separate encryption and authentication mechanisms.
Instead, our implementation employs a combined \acrshort{aead} scheme (more specifically, AES-256 in CCM mode).
This uses a single key, derived by hashing the curve point, to ensure confidentiality and integrity of the data.

The implementation includes the fix proposed in \cite{tan_enhancement_2019}, for which an additional \acrshort{prf} is used to randomize the value of the $\text{index}(\cdot)$ function for nodes of the \gls{access-tree}.
For this, instead of $\text{index}(\cdot)$, the modified $\text{index}'(\cdot) = \text{PRF}(r_l, index(\cdot))$ is used~\cite{tan_enhancement_2019}.
$r_l$ is a random seed value that differs for each layer $l$ of the \gls{access-tree}~\cite{tan_enhancement_2019}.
In our implementation, HMAC-SHA3-512 is used as the \acrshort{prf}.\\

Let $\mathbb{G}$ be a group of order $q$. The four algorithms of the YCT scheme are defined as follows: \\

\emph{Setup}~\cite{yao_lightweight_2015}.
The attribute universe is defined as $\text{U} = \{1, 2, \dots, n\}$ and is fixed.

For every attribute $i \in U$, choose uniformly at random a secret number $s_i \in \mathbb{Z}_q^*$.
Then the public key of attribute $i$ is $P_i = s_i \cdot G$ (i.e. a curve point).

Also, choose uniformly at random the master private key $s \in \mathbb{Z}_q^*$, from which the master public key $PK = s \cdot G$ is derived.

Publish $Params=(PK, P_1, \dots, P_n)$ as the public parameters, privately save $MK = (s, s_1, \dots, s_n)$ as the private master key.
\\

\emph{Encrypt(m, $\omega$, Params)}~\cite{yao_lightweight_2015}.
Input: Message $m$, set of attributes $\omega$ and public parameters $Params$.

Randomly choose $k \in \mathbb{Z}_q^*$ and compute $C' = k \cdot PK$. If $C' = \mathcal{O}$, repeat until $C' \neq \mathcal{O}$.
$C' = (k_x, k_y)$ are the coordinates of the point $C'$. $k_x$ is used as the encryption key and $k_y$ as the integrity key.

Then compute $C_i = k \cdot P_i$ for all attributes $i \in \omega$.

Encrypt the actual message as $c = \text{Enc}(m, k_x)$, generate a Message Authentication Code $\text{mac}_m = \text{HMAC}(m, k_y)$.

Return the ciphertext $CM = (\omega, c, \text{mac}_m, \{C_i | i \in \omega\})$\\

\emph{KeyGen($\Gamma$, MK)}~\cite{yao_lightweight_2015}.
Input: \glspl{access-tree} $\Gamma$ and master key $MK$.

For each layer $l = 0, 1, \dots$ of the \gls{access-tree}, generate a random seed value $r_l \in \mathcal{K}_{PRF}$ from the PRF's key space.

For each node $u$ in the \gls{access-tree} $\Gamma$, recursively define polynomials $q_u(x)$ with degree $(d_u - 1)$, starting from the root.

For the root $r$, set $q_r(0) = s$ and randomly choose $(d_r - 1)$ other points to determine the polynomial $q_r(x)$.
Then, for any other node $u$ (including leafs), set $q_u(0) = q_{\text{parent}(u)}(\text{index}'(u))$ and choose $(d_u -1)$ other points for $q_u$, similar to above.

Whenever $u$ is a leaf node, use $q_u(x)$ to define a secret share $D_u = \frac{q_u(0)}{s_i}$; where $i = \text{attr}(u)$, $s_i$ the randomly chosen secret number from \emph{Setup} and $s_i^{-1}$ the inverse of $s_i$ in $\mathbb{Z}_q^*$.

Return the generated key as $D = (\{D_u | u \text{ leaf node of } \Gamma\}, \{r_0, r_1, \dots \})$.
\\

\emph{Decrypt(CM, D, Params)}~\cite{yao_lightweight_2015}. Input: Ciphertext $CM$, decryption key $D$ and public parameters $Params$.

Decryption is split into two phases: Reconstructing the curve point $C'$ to get the encryption and integrity keys, and actual decryption of the ciphertext.

First, define a recursive decryption procedure for a node $u$: $\text{DecryptNode}(CM, D, u)$. \\
For leaf nodes with $i = \text{attr}(u)$:
\begin{equation*}
    \text{DecryptNode}(CM, D, u) = \begin{cases}
        D_u \cdot C_i \stackrel{(*)}{=} q_u(0) \cdot k \cdot G & i \in \omega\\
        \perp & i \notin \omega
    \end{cases}\\
\end{equation*}

Where the equality $(*)$ holds because $s_i$ and $s_i^{-1}$ cancel out: 
\begin{equation*}
    D_u \cdot C_i = q_u(0) \cdot s_i^{-1} \cdot k \cdot P_i = q_u(0) \cdot s_i^{-1} \cdot k \cdot s_i \cdot G = q_u(0) \cdot k \cdot G
\end{equation*}

For an internal node $u$ on layer $l$, call $\text{DecryptNode}(CM, D, v)$ for each of its childen $v$. If for less than $d_u$ of the child nodes $\text{DecryptNode}(CM, D, v) \neq \perp$, return $\text{DecryptNode}(CM, D, )=\perp$.
Then let $\omega_u$ be an arbitrary subset of $d_u$ child nodes of $u$, where for all $v \in \omega_u$, $\text{DecryptNode}(CM, D, v) \neq \perp$.
Then $\text{DecryptNode}(CM, D, u)$ is defined as follows, where $i = \text{index}(v)$, $\omega'_u = \{\text{index}(v) | v \in \omega_u\}$.
\begin{equation*}
    \begin{split}
        &~\text{DecryptNode}(CM, D, u)\\
        =& \sum_{v \in \omega_u} \Delta_{\omega'_u, i}(0) \cdot \text{DecryptNode}(CM, D, v)\\
        =& \sum_{v \in \omega_u} \Delta_{\omega'_u, i}(0) \cdot q_v(0) \cdot k \cdot G\\
        =& \sum_{v \in \omega_u} \Delta_{\omega'_u, i}(0) \cdot q_{\text{parent}(v)}(\text{index}(v)) \cdot k \cdot G\\
        =& \sum_{v \in \omega_u} \Delta_{\omega'_u, i}(0) \cdot q_u(i) \cdot k \cdot G\\
        \stackrel{(*)}{=}&~q_u(0) \cdot k \cdot G
    \end{split}
\end{equation*}

The equality $(*)$ holds because $\sum_{v \in \omega'_u} \Delta_{\omega'_u, i}(0) \cdot q_u(i) = q_u(0)$ is exactly the lagrange interpolation polynomial $q_u(x)$ at $x = 0$ with respect to the points $\{(index(v), q_v(0)) | v \in \omega_u\}$. 

This means for the root $r$ of the \gls{access-tree} $\Gamma$, we have
\begin{equation*}
    \text{DecryptNode}(CM, D, r) =  q_r(0) \cdot k \cdot G = s \cdot k \cdot G = (k'_x, k'_y)
\end{equation*}

With $k'_x$ the decryption key for $m$ and $k'_y$ the integrity key. Therefore now decrypt $m' = \text{Dec}(c, k'_x)$.

Now check if $\text{HMAC}(m', k'_y) = \text{mac}_m$. If yes, the ciphertext has been correctly decrypted and was not tampered with. Return $m'$, otherwise return $\perp$.