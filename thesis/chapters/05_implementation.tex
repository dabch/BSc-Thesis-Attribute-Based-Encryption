\chapter{Implementation}

\section{Building Blocks} % TODO rename this

\subsection{Hardware}

The main goal of this project was to implement an \acrshort{abe} scheme on a constrained embedded ARM processor.
More specifically, the chip used was a Nordic Semiconductor nRF52840 with a 64MHz Cortex M4 CPU, 256kB of RAM and 1MB of flash storage.
For the detailed specifications, see \cite{noauthor_nrf52840_nodate}.
This SoC will be referred to simply as `the SoC'.

For reference, the implementation was also tested on a standard laptop, referred to as `the Laptop'.
More specifically, this system has a 2.7GHz Intel i7-7500U CPU and 16GB of RAM.
It runs a Linux-based operating system.

% The implementation of \acrshort{abe} on the SoC in Rust posed three main challenges:
% \begin{itemize}
%     \item The availability of pairing libraries in Rust is very limited.
%     \item Due to the lack of an operating system, there is no standard library, no dynamic memory allocation and no easy access to a \acrshort{csprng}.
%     \item The low speed of computation compared to PCs or even other \gls{iot} platforms (e.g. Raspberry Pi).
% \end{itemize}

\subsection{Programming language and libraries}

\subsubsection{Rust}

Rust was chosen as programming language for this project.
First, it is a compiled language and thus incurs little overhead at runtime. Its speed is comparable to that of C/C++. % citation needed for that last bit
Second, it provides much stronger memory safety guarantees than other compiled languages (especially C/C++ where extreme care is required to avoid introducing exploitable vulnerabilities).
This is especially attractive for security-critical components like an encryption library.

\subsubsection{The \texttt{rabe-bn} library}

A Rust-only implementation of elliptic curves and a pairing is provided by the open-source library \texttt{rabe-bn}\footnote{\url{https://github.com/georgbramm/rabe-bn}}, a derivative of the \texttt{bn} library by Zcash~\cite{bowe_bn_2016}.
It implements a concrete pairing on 256-bit \emph{BN curves}. % TODO security level?!
BN curves are a family of pairing-friendly elliptic curves proposed by Barreto and Naehrig \cite{barreto_pairing-friendly_2006}.

The 256-bit modulus of the BN curve used in \texttt{rabe-bn} was originally believed to provide a security level of 128 bits~\cite{ben-sasson_succinct_2013}. 
Due to the discovery of better attacks on the underlying cryptographic assumptions, the estimate for the security level has been revised down to 100 bits~\cite{yonezawa_pairing-friendly_2019}.

The library provides four structs: \texttt{G1}, \texttt{G2} and \texttt{Gt}, elements of the groups $\mathbb{G}_1$, $\mathbb{G}_2$ and $\mathbb{G}_T$, respectively.
Let their orders be $r$, then \texttt{Fr} represents an element of the field $\mathbb{F}_r$.

For the elliptic curve groups (structs \texttt{G1} and \texttt{G2}), additive notation is used and the \texttt{*} operator is conveniently overloaded.
The target group (struct \texttt{Gt}) uses multiplicative notation.
For this reason, the description of the schemes in Chapter~\ref{chapter:constructions} has also been adapted to use compatible notation.

\subsubsection{nRF52840 HAL crate}
For easier access to the peripherals of the SoC, the \acrfull{hal} \gls{crate} \texttt{nrf52840-hal} was also used.
It provides simplified access to the hardware \acrfull{rng} and the timers.
Strictly speaking, these were not necessary to build a Rust library (timers are only needed for evaluation and the library interface allows the caller to pass their desired random number generator).
However, for testing and actual use of the library bare-metal on the SoC, both the \acrshort{rng} and the timers were needed.

\subsubsection{\texttt{heapless} crate}
The \texttt{heapless}\footnote{\url{https://crates.io/crates/heapless}} crate provides stack-allocated versions of some of the data structures from \texttt{std::collections}. 
Most important were \texttt{heapless::Vec} (replaces \texttt{std::vec::Vec}) and \texttt{heapless::IndexMap} (replaces \texttt{std::collections::HashMap}).
These data structures are statically allocated and expect their desired capacity as an additional generic type parameter.

\section{Porting \texttt{rabe-bn} to the SoC}

The implementation of \texttt{rabe-bn} unfortunately relied on the standard library (mostly through the use of heap-allocated dynamic vectors, i.e. \texttt{std::vec::Vec}) and is thus not suited for bare-metal applications.
Rust provides the dependency-free and platform-agnostic \texttt{core} library as an alternative to the standard library.
This library does not depend on an operating system or dynamic memory allocation, and thus does not include heap-allocated data structures (like \texttt{std::vec::Vec}).

Therefore, I rewrote the \texttt{rabe-bn} library to introduce a cargo-feature \texttt{std} which controls the inclusion of the standard library and is enabled by default.
If this feature is disabled, the \texttt{core} library and stack-allocated collections of fixed size from the \texttt{heapless} crate are used instead.

Some further modifications were necessary to implement the \texttt{core::fmt::Display} trait for the \texttt{Gt} struct in a bare-metal compatible manner.
The implementation of this trait was used in conjunction with \acrshort{sha3} as a \gls{kdf} to create an AES key from curve points.
The behavior of the \texttt{core::fmt::Display} implementation stayed exactly the same to ensure interoperability with the original \texttt{rabe-bn} library.

With these modifications, the \texttt{rabe-bn} library runs on the SoC.

\section{Random Number Generation} % TODO maybe adapt this to use a CSPRNG to speed up stuff on the SoC
Regular Rust programs use the \texttt{rand} \gls{crate}'s \verb+ThreadRng+ struct to generate random numbers.
\texttt{ThreadRng} is cryptographically secure~\cite{noauthor_rust_nodate}, but it relies on the operating system randomness pool for seeding.

Therefore, this generator is unavailable on the SoC.
Instead we use the hardware \acrshort{rng}.
The \texttt{nrf52840-hal} crate directly implements the trait \texttt{rand::RngCore} for the hardware \acrshort{rng}, which makes it extremely easy to use.
This generator, however, is quite slow and speed can differ greatly: With bias correction enabled (required for uniform distribution of the generated data), typically around 120$\mu$s per byte~\cite{noauthor_nrf52840_nodate}.

To alleviate this, the hardware \acrshort{rng} is only used to seed a ChaCha20 pseudorandom number generator (crate \texttt{rand\_chacha}).
This is essentially the same construction as the current implementation of \texttt{ThreadRng}~\cite{noauthor_rust_nodate}.

% TODO for evaluation: seed a CSPRNG with a seed from the TRNG!

\section{Representation of Access Trees}
The Rust type system is very well suited to represent the type of tree structures we need for Access Trees.
A simple implementation might look like the one in Listing~\ref{lst:access-tree-naive}. 

\begin{lstlisting}[caption={Simple Implementation of Access Trees (using the standard library)},label={lst:access-tree-naive}]
enum AccessTree<'a> {
    // threshold, vector of children
    Node(u64, std::vec::Vec<AccessTree<'a>>),
    // reference to the attribute label
    Leaf(&'a str),
}
\end{lstlisting}

This, however, does not work when the \texttt{std::vec::Vec} is replaced by a stack-allocated \texttt{heapless::Vec}:
The \texttt{std::vec::Vec} is allocated on the heap and thus only a pointer to the vector needs to be stored in the \texttt{AccessTreeNode}.
This pointer has constant size, of course.

A \texttt{heapless::Vec} is not located on the heap, but directly inside the \texttt{AccessTreeNode}.
Even if there is a limit on the number of children a single inner node might have, there is no limit to the depth of the access tree.
Therefore, an \texttt{AccessTreeNode} might be arbitrarily large because the \texttt{heapless::Vec} might need to hold an arbitrary number of child nodes.

Because of this, Access Trees were implemented as a flat slice of nodes as in Listing~\ref{lst:access-tree-heapless}. 
The vector of children doesn't hold references to the children themselves, but only their index within the vector of Access Tree nodes.
This again introduces an indirection (like the heap pointer in the simple implementation) and allows the enums to have constant size.

\begin{lstlisting}[caption={Refined implementation of Access Trees (works without standard library)},label={lst:access-tree-heapless}]
type AccessTree<'a, 'b> = &'b [AccessNode<'a>];
enum AccessNode<'a> {
    // threshold, vector of child indexes
    Node(u64, heapless::Vec<u8, consts::U16>),
    // reference to the attribute label
    Leaf(&'a str),
}
\end{lstlisting}