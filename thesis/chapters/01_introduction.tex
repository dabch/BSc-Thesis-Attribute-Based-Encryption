% !TeX root = ../main.tex
% Add the above to each chapter to make compiling the PDF easier in some editors.

\chapter{Introduction}\label{chapter:introduction}

% TODO starting sentence

For an example use case of such a library, see Figure~\ref{fig:system-architecture}.
The patient receives the medical sensor from their doctor (e.g. embedded in a T-shirt). 
Because the sensor must have a small footprint and run on battery for an extended time period, it is not able to communicate via the internet directly.
Instead, the user installs an app on their smartphone which acts as a gateway.
This app receives the data from the sensor via \acrfull{ble} and uses the phone's internet connection to upload it to the cloud.
Authorized users (e.g. the patient themselves or their doctor) can then download the data from the cloud.

Since the collected data in this system (e.g. the ECG) is related to the patient's health, it must be protected with particular care:
For example, the \acrshort{gdpr} lists health data as a \emph{special category} of data that enjoys extra protections compared to regular personal data. % TODO cite this from somewhere
This is why simple transport encryption (e.g. by \acrshort{ble} between the sensor and gateway and TLS between the gateway and cloud) is not sufficient: It does not protect the data from a malicious gateway or cloud provider.
Given the strong security requirements regarding medical data, we require end-to-end encryption, i.e. the data is encrypted at the point of origin (the sensor) and decrypted only locally when accessed by an authorized user (e.g. on the attending doctor's computer).
The data is encrypted during transmission (both via \acrshort{ble} and over the internet) and storage on the cloud server.
Neither the gateway nor the cloud provider must be trusted.


~\\
~\\
If the gateway is simply a smartphone, it may be compromised by malicious actors

% section: motivation
The sensor might for example measure an ECG (electrocardiogram), the heart rate or the blood sugar of the wearer.
This information is highly sensitive and requires careful protection.
To eliminate trust in the 


\begin{figure}[t]
    \centering
    \begin{tikzpicture}[fullnode/.style={draw, minimum width=3.75cm, minimum height=1.5cm, align=center}]
        % \draw (0, 0) node (soc) {Sensor\\(nRF52840 SoC)} |- ++(5, 0) node (gateway) {Gateway\\(Smartphone or Raspberry Pi)};
        \node[fullnode] (soc) {Sensor \\(nRF52840 SoC)};
        \node[fullnode] (gateway) [right=4cm of soc] {Gateway\\(e.g. Smartphone)};
        \node[fullnode] (cloud) [below=2.5cm of gateway] {Cloud\\(e.g. AWS instance)};
        \node[fullnode] (kgc) [below=2.5cm of soc] {Key Generation\\Center};

        \draw[->] (soc) -- (gateway) node [pos=0.5, align=center] {Bluetooth\\Low Energy};
        \draw[<->] (gateway) -- (cloud) node [pos=0.5, align=center] {Internet\\(TCP/IP)};

        \node[fullnode] (user1) [below=2.5cm of kgc] {User 1\\(e.g. Patient)};
        \node[fullnode] (user2) [below=2.5cm of cloud] {User 2\\(e.g. Doctor)};

        % \draw[->] (kgc) -- (gateway) node [pos=0.5] {Internet};
        \draw[->, dotted] (kgc) -- (soc);% node [pos=0.5, align=center] {Key exchanged\\before deployment};
        \draw[->, dotted] (kgc) -- (user1);% node [pos=0.5] {Internet};
        \draw[->, dotted] (kgc) -- (user2);% node [pos=0.5] {Internet};
        \draw[<->] (user1) -- (cloud) node [pos=0.25, sloped, above] {Internet};
        \draw[<->] (user2) -- (cloud) node [pos=0.5] {Internet};

        \node [draw, dashed, fit=(soc) (gateway), inner sep=3mm, label={[anchor=south west]north west:Carried by the patient}] {};
        \node [red, draw, dashed, fit=(gateway) (cloud), inner sep=5mm, label={[red,anchor=south]above:Not trusted}] {};
    \end{tikzpicture}
    \caption{Simplified system architecture for end-to-end Attribute Based Encryption}
    \label{fig:system-architecture}
\end{figure}

\emph{Martin: }
hier wünsche ich mir vor allem die Motivation und eine Einordnung ins große Ganze. Du kannst gern den Medisec Anwendungsfall als Beispiel hernehmen, an dem du das diskutierst, musst du aber nicht

- Warum sollte ich ABE hernehmen?\\ 
- Welche Probleme löst es, die ich sonst nicht elegant lösen kann? (Update nach dem Lesen von 2.1: da erklärst du es super. Dann hier halt in kurz "gut für Verschlüsselung an mehrere Empfänger") \\
- Welche Alternativen zu ABE gäbe es denn überhaupt? Was ist nervig an ABE (z.B. dass man ein KGC braucht?)\\
- Ist das sinnvoll, das auf Mikrocontrollern zu machen? \\

In der BA ist es noch nicht so wichtig wie in der MA, dass du eine zentrale Forschungsfrage hinschreibst. Wenn du es aber kannst, macht es den Rest leichter weil du die ganze Arbeit dran strukturieren kannst. Vor allem beim Related Work zusammenstellen hilft es, siehe meine Anmerkungen da. 
Nach meinem Verständnis behandelst du die Frage "kann man ABE gescheit auf Mikrocontrollern machen?". Ggf. kann man den Titel der BA noch dahingehend anpassen, dass er mehr sciency klingt ("Implementierung" ist "Ingenieurs-Handwerk" und das findet der Academia Mensch unter seiner Würde. Ist Quatsch, aber leider ticken die so). Also wenn du mehr sciency klingen willst etwa "Evaluating the feasibility of a Rust-based ABE Library on MCUs" --> Evaluation ist wieder die ureigenste Aufgabe des Scientisten, also alles gut.


% \section{Section}
% Citation test~\parencite{latex}.

% \subsection{Subsection}

% See~\autoref{tab:sample}, \autoref{fig:sample-drawing}, \autoref{fig:sample-plot}, \autoref{fig:sample-listing}.

% \begin{table}[htpb]
%   \caption[Example table]{An example for a simple table.}\label{tab:sample}
%   \centering
%   \begin{tabular}{l l l l}
%     \toprule
%       A & B & C & D \\
%     \midrule
%       1 & 2 & 1 & 2 \\
%       2 & 3 & 2 & 3 \\
%     \bottomrule
%   \end{tabular}
% \end{table}

% \begin{figure}[htpb]
%   \centering
%   % This should probably go into a file in figures/
%   \begin{tikzpicture}[node distance=3cm]
%     \node (R0) {$R_1$};
%     \node (R1) [right of=R0] {$R_2$};
%     \node (R2) [below of=R1] {$R_4$};
%     \node (R3) [below of=R0] {$R_3$};
%     \node (R4) [right of=R1] {$R_5$};

%     \path[every node]
%       (R0) edge (R1)
%       (R0) edge (R3)
%       (R3) edge (R2)
%       (R2) edge (R1)
%       (R1) edge (R4);
%   \end{tikzpicture}
%   \caption[Example drawing]{An example for a simple drawing.}\label{fig:sample-drawing}
% \end{figure}

% \begin{figure}[htpb]
%   \centering

%   \pgfplotstableset{col sep=&, row sep=\\}
%   % This should probably go into a file in data/
%   \pgfplotstableread{
%     a & b    \\
%     1 & 1000 \\
%     2 & 1500 \\
%     3 & 1600 \\
%   }\exampleA
%   \pgfplotstableread{
%     a & b    \\
%     1 & 1200 \\
%     2 & 800 \\
%     3 & 1400 \\
%   }\exampleB
%   % This should probably go into a file in figures/
%   \begin{tikzpicture}
%     \begin{axis}[
%         ymin=0,
%         legend style={legend pos=south east},
%         grid,
%         thick,
%         ylabel=Y,
%         xlabel=X
%       ]
%       \addplot table[x=a, y=b]{\exampleA};
%       \addlegendentry{Example A};
%       \addplot table[x=a, y=b]{\exampleB};
%       \addlegendentry{Example B};
%     \end{axis}
%   \end{tikzpicture}
%   \caption[Example plot]{An example for a simple plot.}\label{fig:sample-plot}
% \end{figure}

% \begin{figure}[htpb]
%   \centering
%   \begin{tabular}{c}
%   \begin{lstlisting}[language=SQL]
%     SELECT * FROM tbl WHERE tbl.str = "str"
%   \end{lstlisting}
%   \end{tabular}
%   \caption[Example listing]{An example for a source code listing.}\label{fig:sample-listing}
% \end{figure}
