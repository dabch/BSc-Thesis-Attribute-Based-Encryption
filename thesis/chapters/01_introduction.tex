% !TeX root = ../main.tex
% Add the above to each chapter to make compiling the PDF easier in some editors.

\chapter{Introduction}\label{chapter:introduction}

% Introduction Kommentare Ehsan:
% - kurze Zusammenfassung der wichtigsten Aspekte des Themas -> welcher Teil davon wird in der Arbeit bearbeitet und warum?
% - Sätze wiederholen sich öfter
% - vielleicht auf Spannungsfeld zwischen Leistung und Sicherheit eingehen
% - ein bis zwei Sätze warum ABE cool ist und besser als Alternativen (z.B. Kerberos-ähnliche symmetrische Krypto)

% TODO starting sentence
% The \acrfull{iot} offers great potential for improved efficiency, comfort and safety in many areas. % TODO find some citations
% Applications of the \acrshort{iot} include household devices (``Smart Home''), infrastructure, manufacturing, healthcare and many other fields.



\acrfull{abe} promises to provide security while improving flexibility over standard cryptosystems.
Its approach of combining \glspl{attribute} and \glslink{access-policy}{policies} describing admissible attribute combinations is closer to traditional notions of access control (e.g. RBAC)~\cite{bethencourt_ciphertext-policy_2007}. % TODO find citation

This increased flexibility is also desirable for \acrshort{iot} applications.
However, the \acrfull{iot} is especially affected by the trade-off between cost and security:
Because hardware must be cheap and has to run on batteries, it is often highly constrained.
Many traditional security measures, even some standard cryptographic schemes, are not practical on \acrshort{iot} devices.
Compared to standard cryptgraphy, most \acrshortpl{abes} demand even more resources~\cite{wang_performance_2014}.
Thus, the feasibility of \acrshort{abe} on constrained \acrshort{iot} nodes is disputed~\cite{wang_performance_2014,ambrosin_feasibility_2016,ambrosin_feasibility_2015,girgenti_feasibility_2019,borgh_attribute-based_2016}.

The objective of this thesis is to assess to what extent \acrshort{abe} can be practically applied on ARM Cortex M4 processors.
To this end, an \acrshort{abe} library is developed using the Rust programming language.
This library is then tested and evaluated on a Cortex M4 SoC.
In addition, this thesis aims to give an easy-to-understand explanation of \acrshort{abe} and how it can be implemented.


% The increasing ubiquity of connected sensors and actuators in the ``things'' around us raises serious concerns about privacy and security.
% For example, an insecure ``Smart Home'' might reveal to burglars that its owners are on vacation (e.g. if they can see that the heating is turned off and no windows have been opened for a few days).
% Even worse, the ``Smart Door Lock'' might even be exploited to let them right in.

% Protecting these networks by means of cryptographic protocols is possible, but despite the possible negative consequences, many \acrshort{iot} applications are still insecure.
% One reason for this is that, due to large numbers of connected devices, \acrshort{iot} hardware must be cheap.
% Therefore, most of the ``things'' participating in the \acrshort{iot} are severely constrained in terms of computational power, memory and storage size and power consumption. 
% This limits the applicability of many modern cryptographic concepts (e.g. signatures and, sometimes, even public-key encryption) because the devices involved are simply too slow and small.

% In recent years, small microcontrollers have become much cheaper and more powerful: 
% Increasingly, 8-bit architectures are replaced by embedded 32-bit architectures such as ARM Cortex even for simple applications. % TODO find citation and check that this is true
% This brings a large increase in computational performance, storage and memory size.
% For the first time, even more advanced cryptographic constructions such as those based on bilinear pairings become feasible even on the smallest of connected sensors.

% The feasibility of one such relatively new type of cryptosystems shall be examined in this thesis: 
% \acrfull{abe} allows very intuitive and fine-grained access control by means of expressive access policies specified at the time of encryption or key issuing.
% With a few exceptions, \acrshort{abe} is based on bilinear pairings, which are quite computationally expensive.
% The feasibility of \acrshort{abe} on constrained devices such as Smartphones, powerful \acrshort{iot} nodes (e.g. Raspberry Pi) and small sensor nodes is disputed in literature~\cite{wang_performance_2014,ambrosin_feasibility_2016,ambrosin_feasibility_2015,girgenti_feasibility_2019,borgh_attribute-based_2016}.

% The goal of this thesis is to evaluate the feasibility of an \acrshort{abe} library for a small 32-bit MCU.
% More specifically, the target system is an nRF52840 SoC with a 64MHz ARM Cortex M4 CPU and 256KB of RAM.
% To this end, such a library is implemented using the Rust programming language and the performance of two \acrshortpl{abes} on the SoC is evaluated.

\begin{figure}[h]
    \centering
    \begin{tikzpicture}[fullnode/.style={draw, minimum width=3.75cm, minimum height=1.25cm, align=center}]
        % \draw (0, 0) node (soc) {Sensor\\(nRF52840 SoC)} |- ++(5, 0) node (gateway) {Gateway\\(Smartphone or Raspberry Pi)};
        \node[fullnode] (soc) {Sensor \\(nRF52840 SoC)};
        \node[fullnode] (gateway) [right=4cm of soc] {Gateway\\(e.g. Smartphone)};
        \node[fullnode] (cloud) [below=1.5cm of gateway] {Cloud\\(e.g. AWS instance)};
        \node[fullnode] (kgc) [below=1.5cm of soc] {Key Generation\\Center};

        \draw[->] (soc) -- (gateway) node [pos=0.5, align=center] {Bluetooth\\Low Energy};
        \draw[<->] (gateway) -- (cloud) node [pos=0.5, align=center] {Internet\\(TCP/IP)};

        \node[fullnode] (user1) [below=1.5cm of kgc] {User 1\\(e.g. Patient)};
        \node[fullnode] (user2) [below=1.5cm of cloud] {User 2\\(e.g. Doctor)};

        % \draw[->] (kgc) -- (gateway) node [pos=0.5] {Internet};
        \draw[->, dotted] (kgc) -- (soc);% node [pos=0.5, align=center] {Key exchanged\\before deployment};
        \draw[->, dotted] (kgc) -- (user1);% node [pos=0.5] {Internet};
        \draw[->, dotted] (kgc) -- (user2);% node [pos=0.5] {Internet};
        \draw[<->] (user1) -- (cloud) node [pos=0.25, sloped, above] {Internet};
        \draw[<->] (user2) -- (cloud) node [pos=0.5] {Internet};

        \node [draw, dashed, fit=(soc) (gateway), inner sep=1mm, label={[anchor=south west]north west:Carried by the patient}] {};
        \node [red, draw, dashed, fit=(gateway) (cloud), inner sep=2mm, label={[red,anchor=south]above:Not trusted}] {};
    \end{tikzpicture}
    \caption[Simplified use case for our ABE library]{Simplified use case for end-to-end Attribute Based Encryption with encryption on a constrained sensor MCU. The ABE library developed for this thesis runs on the sensor.}
    \label{fig:system-architecture}
\end{figure}

For a real-world use-case of this library in the medical field, see Figure~\ref{fig:system-architecture}.
It is assumed that the sensor (e.g. an ECG or blood glucose sensor) is unable communicate with the internet directly.
Therefore, the data is sent to a gateway by \acrfull{ble} and then uploaded to the cloud.

The goal of this project is to enable end-to-end \acrlong{abe}:
Measurements are encrypted on the sensor before they are transmitted over \acrshort{ble} and only decrypted when read by an authorized client (e.g. the attending doctor).
In this system, the sensor exclusively needs to encrypt data; decryption, key generation and setup are not required on the constrained node.
I believe that this is the case in many \acrshort{iot} applications.

The end-to-end \acrshort{abe} approach does not require trust in the gateway or cloud provider.
I believe that trusting the gateway or cloud provider would be an unreasonable assumption in many scenarios.

An alternative to encryption with \acrshort{abe} on the sensor would be outsourcing the \acrshort{abe} operations to the gateway:
The sensor and gateway share a secret key, which is used to secure the communication over \acrshort{ble}.
The gateway then decrypts the symmetrically encrypted data and re-encrypts it using \acrshort{abe}.
This approach does not require \acrshort{abe} on the sensor, but only on the gateway.
The latter is assumed to be much more powerful than the sensor.
However, the gateway must be trusted in this scenario.

An alternative to \acrshort{abe} itself would be to let the cloud server enforce access control:
All participants exchange secret keys with the cloud server.
The sensor then uses this key to encrypt the data, which only the cloud server can read.
When a user wants to read some data, they request it at the cloud server.
If the cloud server grants their request, it re-encrypts the data with the respective user's key and transmits it to them.
This approach removes trust in the gateway and requires only symmetric cryptography, which is much faster than \acrshort{abe} or \glspl{pkes}.
The disadvantage is that the cloud server knows all data in plain text. It thus needs to be trusted and represents a single point of failure.

End-to-end \acrshort{abe} does not, however, remove the single point of failure:
A malicious \acrfull{kgc} could issue keys that allow decryption of arbitrary data (see Section~\ref{sec:kgc}).
However, the \acrfull{kgc} is only required once to setup the \acrshort{abe} system and issue the participant's keys.
It is not involved in any encryption or decryption operations.
The \acrshort{kgc} does not need to be online during the use of the system and could e.g. be located on an air-gapped network. 
This greatly reduces the attack surface.
~\\

% For an example use case of such a library, see Figure~\ref{fig:system-architecture}.
% The patient receives a medical sensor from their doctor (e.g. embedded in a T-shirt). 
% Because the sensor must have a small footprint and run on battery for an extended time period, it is not able to communicate via the internet directly.
% Instead, the user installs an app on their smartphone which acts as a gateway.
% This app receives the data from the sensor via \acrfull{ble} and uses the phone's internet connection to upload it to the cloud.
% Authorized users (e.g. the patient themselves or their doctor) can then download the data from the cloud.

% Since the collected data in this system (e.g. the ECG) is related to the patient's health, it must be protected with particular care:
% For example, the \acrshort{gdpr} lists health data as a \emph{special category} of data that enjoys extra protections compared to regular personal data. % TODO cite this from somewhere
% This is why simple transport encryption (e.g. by \acrshort{ble} between the sensor and gateway and TLS between the gateway and cloud) is not sufficient: It does not protect the data from a malicious gateway or cloud provider.
% Given the strong security requirements regarding medical data, we require end-to-end encryption, i.e. the data is encrypted at the point of origin (the sensor) and decrypted only locally when accessed by an authorized user (e.g. on the attending doctor's computer).
% The data is encrypted during transmission (both via \acrshort{ble} and over the internet) and storage on the cloud server.
% Neither the gateway nor the cloud provider need to be trusted. 
% To achieve this, the encryption must be performed on the constrained MCU running the sensor.

% In this use case, the fine-granular access control enabled by \acrshort{abe} is particularly interesting:
% If the system were using a regular \gls{privkes}, when seeing a new doctor, the patient would have to securely share with them their private key.
% This would allow the new doctor to see all the patient's medical information (assuming there is only one key) and provide no access control.

% With a regular \gls{pkes}, securely sharing the key would no longer be a problem. 
% However, the system would have to know the public keys of all present and future doctors when encrypting the data. 
% Again, all data encrypted with a doctor's public key will be visible to them; there are no further access control mechanisms.

% Through the mechanisms more closely described in Section~\ref{sec:background-abe}, KP-\acrshort{abe} allows much more fine-grained access control:
% At the time of encryption, data is described using certain \glspl{attribute}.
% The participants are then issued keys by the \acrfull{kgc}, which allow them to decrypt a piece of data if and only if their key matches the attributes specified during encryption.

% In this use case, only the encryption operation needs to be performed on the constrained MCU.
% As it will turn out, decryption requires much higher performance than encryption.
% Thus, implementing only encryption on the MCU is probably sufficient for many use cases:
% Oftentimes, devices that need to receive (i.e. decrypt) data are connected to some kind of actuator or user interface. 
% In most networks, there are fewer of these nodes than there are sensor nodes, and thus these tend to be more powerful. % TODO nachschauen ob das totaler Stuss ist

% As such, I will be evaluating both encryption and decryption, but focus on the former.
% The thesis is considered a success if decryption can be performed reasonably quickly on the MCU, even if decryption is much more constrained.
% Using \acrshort{abe} -
% ~\\
% ~\\
% If the gateway is simply a smartphone, it may be compromised by malicious actors

% % section: motivation
% The sensor might for example measure an ECG (electrocardiogram), the heart rate or the blood sugar of the wearer.
% This information is highly sensitive and requires careful protection.
% To eliminate trust in the 


~\\

\emph{Martin: }
hier wünsche ich mir vor allem die Motivation und eine Einordnung ins große Ganze. Du kannst gern den Medisec Anwendungsfall als Beispiel hernehmen, an dem du das diskutierst, musst du aber nicht

- Warum sollte ich ABE hernehmen?\\ 
- Welche Probleme löst es, die ich sonst nicht elegant lösen kann? (Update nach dem Lesen von 2.1: da erklärst du es super. Dann hier halt in kurz "gut für Verschlüsselung an mehrere Empfänger") \\
- Welche Alternativen zu ABE gäbe es denn überhaupt? Was ist nervig an ABE (z.B. dass man ein KGC braucht?)\\
- Ist das sinnvoll, das auf Mikrocontrollern zu machen? \\

In der BA ist es noch nicht so wichtig wie in der MA, dass du eine zentrale Forschungsfrage hinschreibst. Wenn du es aber kannst, macht es den Rest leichter weil du die ganze Arbeit dran strukturieren kannst. Vor allem beim Related Work zusammenstellen hilft es, siehe meine Anmerkungen da. 
Nach meinem Verständnis behandelst du die Frage "kann man ABE gescheit auf Mikrocontrollern machen?". Ggf. kann man den Titel der BA noch dahingehend anpassen, dass er mehr sciency klingt ("Implementierung" ist "Ingenieurs-Handwerk" und das findet der Academia Mensch unter seiner Würde. Ist Quatsch, aber leider ticken die so). Also wenn du mehr sciency klingen willst etwa "Evaluating the feasibility of a Rust-based ABE Library on MCUs" --> Evaluation ist wieder die ureigenste Aufgabe des Scientisten, also alles gut.


% \section{Section}
% Citation test~\parencite{latex}.

% \subsection{Subsection}

% See~\autoref{tab:sample}, \autoref{fig:sample-drawing}, \autoref{fig:sample-plot}, \autoref{fig:sample-listing}.

% \begin{table}[htpb]
%   \caption[Example table]{An example for a simple table.}\label{tab:sample}
%   \centering
%   \begin{tabular}{l l l l}
%     \toprule
%       A & B & C & D \\
%     \midrule
%       1 & 2 & 1 & 2 \\
%       2 & 3 & 2 & 3 \\
%     \bottomrule
%   \end{tabular}
% \end{table}

% \begin{figure}[htpb]
%   \centering
%   % This should probably go into a file in figures/
%   \begin{tikzpicture}[node distance=3cm]
%     \node (R0) {$R_1$};
%     \node (R1) [right of=R0] {$R_2$};
%     \node (R2) [below of=R1] {$R_4$};
%     \node (R3) [below of=R0] {$R_3$};
%     \node (R4) [right of=R1] {$R_5$};

%     \path[every node]
%       (R0) edge (R1)
%       (R0) edge (R3)
%       (R3) edge (R2)
%       (R2) edge (R1)
%       (R1) edge (R4);
%   \end{tikzpicture}
%   \caption[Example drawing]{An example for a simple drawing.}\label{fig:sample-drawing}
% \end{figure}

% \begin{figure}[htpb]
%   \centering

%   \pgfplotstableset{col sep=&, row sep=\\}
%   % This should probably go into a file in data/
%   \pgfplotstableread{
%     a & b    \\
%     1 & 1000 \\
%     2 & 1500 \\
%     3 & 1600 \\
%   }\exampleA
%   \pgfplotstableread{
%     a & b    \\
%     1 & 1200 \\
%     2 & 800 \\
%     3 & 1400 \\
%   }\exampleB
%   % This should probably go into a file in figures/
%   \begin{tikzpicture}
%     \begin{axis}[
%         ymin=0,
%         legend style={legend pos=south east},
%         grid,
%         thick,
%         ylabel=Y,
%         xlabel=X
%       ]
%       \addplot table[x=a, y=b]{\exampleA};
%       \addlegendentry{Example A};
%       \addplot table[x=a, y=b]{\exampleB};
%       \addlegendentry{Example B};
%     \end{axis}
%   \end{tikzpicture}
%   \caption[Example plot]{An example for a simple plot.}\label{fig:sample-plot}
% \end{figure}

% \begin{figure}[htpb]
%   \centering
%   \begin{tabular}{c}
%   \begin{lstlisting}[language=SQL]
%     SELECT * FROM tbl WHERE tbl.str = "str"
%   \end{lstlisting}
%   \end{tabular}
%   \caption[Example listing]{An example for a source code listing.}\label{fig:sample-listing}
% \end{figure}
