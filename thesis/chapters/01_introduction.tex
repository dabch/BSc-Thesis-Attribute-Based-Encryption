% !TeX root = ../main.tex
% Add the above to each chapter to make compiling the PDF easier in some editors.

\chapter{Introduction}\label{chapter:introduction}


\emph{Martin: }
hier wünsche ich mir vor allem die Motivation und eine Einordnung ins große Ganze. Du kannst gern den Medisec Anwendungsfall als Beispiel hernehmen, an dem du das diskutierst, musst du aber nicht

- Warum sollte ich ABE hernehmen? 
- Welche Probleme löst es, die ich sonst nicht elegant lösen kann? (Update nach dem Lesen von 2.1: da erklärst du es super. Dann hier halt in kurz "gut für Verschlüsselung an mehrere Empfänger") 
- Welche Alternativen zu ABE gäbe es denn überhaupt? Was ist nervig an ABE (z.B. dass man ein KGC braucht?)
- Ist das sinnvoll, das auf Mikrocontrollern zu machen? 

In der BA ist es noch nicht so wichtig wie in der MA, dass du eine zentrale Forschungsfrage hinschreibst. Wenn du es aber kannst, macht es den Rest leichter weil du die ganze Arbeit dran strukturieren kannst. Vor allem beim Related Work zusammenstellen hilft es, siehe meine Anmerkungen da. 
Nach meinem Verständnis behandelst du die Frage "kann man ABE gescheit auf Mikrocontrollern machen?". Ggf. kann man den Titel der BA noch dahingehend anpassen, dass er mehr sciency klingt ("Implementierung" ist "Ingenieurs-Handwerk" und das findet der Academia Mensch unter seiner Würde. Ist Quatsch, aber leider ticken die so). Also wenn du mehr sciency klingen willst etwa "Evaluating the feasibility of a Rust-based ABE Library on MCUs" --> Evaluation ist wieder die ureigenste Aufgabe des Scientisten, also alles gut.


% \section{Section}
% Citation test~\parencite{latex}.

% \subsection{Subsection}

% See~\autoref{tab:sample}, \autoref{fig:sample-drawing}, \autoref{fig:sample-plot}, \autoref{fig:sample-listing}.

% \begin{table}[htpb]
%   \caption[Example table]{An example for a simple table.}\label{tab:sample}
%   \centering
%   \begin{tabular}{l l l l}
%     \toprule
%       A & B & C & D \\
%     \midrule
%       1 & 2 & 1 & 2 \\
%       2 & 3 & 2 & 3 \\
%     \bottomrule
%   \end{tabular}
% \end{table}

% \begin{figure}[htpb]
%   \centering
%   % This should probably go into a file in figures/
%   \begin{tikzpicture}[node distance=3cm]
%     \node (R0) {$R_1$};
%     \node (R1) [right of=R0] {$R_2$};
%     \node (R2) [below of=R1] {$R_4$};
%     \node (R3) [below of=R0] {$R_3$};
%     \node (R4) [right of=R1] {$R_5$};

%     \path[every node]
%       (R0) edge (R1)
%       (R0) edge (R3)
%       (R3) edge (R2)
%       (R2) edge (R1)
%       (R1) edge (R4);
%   \end{tikzpicture}
%   \caption[Example drawing]{An example for a simple drawing.}\label{fig:sample-drawing}
% \end{figure}

% \begin{figure}[htpb]
%   \centering

%   \pgfplotstableset{col sep=&, row sep=\\}
%   % This should probably go into a file in data/
%   \pgfplotstableread{
%     a & b    \\
%     1 & 1000 \\
%     2 & 1500 \\
%     3 & 1600 \\
%   }\exampleA
%   \pgfplotstableread{
%     a & b    \\
%     1 & 1200 \\
%     2 & 800 \\
%     3 & 1400 \\
%   }\exampleB
%   % This should probably go into a file in figures/
%   \begin{tikzpicture}
%     \begin{axis}[
%         ymin=0,
%         legend style={legend pos=south east},
%         grid,
%         thick,
%         ylabel=Y,
%         xlabel=X
%       ]
%       \addplot table[x=a, y=b]{\exampleA};
%       \addlegendentry{Example A};
%       \addplot table[x=a, y=b]{\exampleB};
%       \addlegendentry{Example B};
%     \end{axis}
%   \end{tikzpicture}
%   \caption[Example plot]{An example for a simple plot.}\label{fig:sample-plot}
% \end{figure}

% \begin{figure}[htpb]
%   \centering
%   \begin{tabular}{c}
%   \begin{lstlisting}[language=SQL]
%     SELECT * FROM tbl WHERE tbl.str = "str"
%   \end{lstlisting}
%   \end{tabular}
%   \caption[Example listing]{An example for a source code listing.}\label{fig:sample-listing}
% \end{figure}
