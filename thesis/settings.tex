\PassOptionsToPackage{table,svgnames,dvipsnames}{xcolor}

\usepackage[utf8]{inputenc}
\usepackage[T1]{fontenc}
\usepackage[sc]{mathpazo}
\usepackage[ngerman,american]{babel}
\usepackage[autostyle]{csquotes}
\usepackage[dvipsnames]{xcolor}
\usepackage[%
  backend=biber,
  url=true,
  urldate=comp,
  style=ieee,
  maxnames=4,
  minnames=3,
  maxbibnames=99,
  giveninits,
  uniquename=init]{biblatex} % TODO: adapt citation style
\usepackage{graphicx}
\usepackage{scrhack} % necessary for listings package
\usepackage{listings}
\usepackage{listings-rust}
\usepackage{lstautogobble}
\usepackage{tikz}
\usepackage{pgfplots}
\usepackage{pgfplotstable}
\usepackage{booktabs}
\usepackage[final]{microtype}
\usepackage{caption}
\usepackage{subcaption}
\usepackage{amsfonts}
\usepackage{amsthm}
\usepackage{amsmath}
\usepackage{mathtools}
\usepackage[hidelinks]{hyperref} % hidelinks removes colored boxes around references and links
\usepackage[acronym]{glossaries}

\usetikzlibrary{intersections}

\def\PolynomialSSS(#1){8 + 7 * #1 - 6 * #1^2 + #1 ^ 3}%

\renewcommand\textbullet{\ensuremath{\bullet}}

\lstset{language=Rust,frame=single}

\theoremstyle{theorem}
\newtheorem{theorem}{Theorem}[chapter]

\theoremstyle{definition}
\newtheorem{definition}{Definition}[chapter]

\bibliography{bibliography}

\setkomafont{disposition}{\normalfont\bfseries} % use serif font for headings
\linespread{1.05} % adjust line spread for mathpazo font

% Add table of contents to PDF bookmarks
\BeforeTOCHead[toc]{{\cleardoublepage\pdfbookmark[0]{\contentsname}{toc}}}

% Define TUM corporate design colors
% Taken from http://portal.mytum.de/corporatedesign/index_print/vorlagen/index_farben
\definecolor{TUMBlue}{HTML}{0065BD}
\definecolor{TUMSecondaryBlue}{HTML}{005293}
\definecolor{TUMSecondaryBlue2}{HTML}{003359}
\definecolor{TUMBlack}{HTML}{000000}
\definecolor{TUMWhite}{HTML}{FFFFFF}
\definecolor{TUMDarkGray}{HTML}{333333}
\definecolor{TUMGray}{HTML}{808080}
\definecolor{TUMLightGray}{HTML}{CCCCC6}
\definecolor{TUMAccentGray}{HTML}{DAD7CB}
\definecolor{TUMAccentOrange}{HTML}{E37222}
\definecolor{TUMAccentGreen}{HTML}{A2AD00}
\definecolor{TUMAccentLightBlue}{HTML}{98C6EA}
\definecolor{TUMAccentBlue}{HTML}{64A0C8}

% Settings for pgfplots
\pgfplotsset{compat=newest}
\pgfplotsset{
  % For available color names, see http://www.latextemplates.com/svgnames-colors
  cycle list={TUMBlue\\TUMAccentOrange\\TUMAccentGreen\\TUMSecondaryBlue2\\TUMDarkGray\\},
}


\makeglossaries


% Settings for lstlistings
\lstset{%
  basicstyle=\ttfamily,
  columns=fullflexible,
  autogobble,
  keywordstyle=\bfseries\color{TUMBlue},
  stringstyle=\color{TUMAccentGreen}
}

\tikzset{
  key/.pic={
    \filldraw (2.7,1) coordinate (-name) -- ++(-7, 0) -- ++(-1,-1) -- ++(1, -1) -- ++(0.5, 0.5) -- ++(0.5, -0.5) -- ++(0.75, 0.75) -- ++(0.75, -0.75) -- ++(0.5, 0.5) -- ++(0.5, -0.5) -- ++(1, 1) -- ++(1, -1) -- ++(0.5, 0.5) -- ++(0.5, -0.5) -- ++(0.5, 0) arc (-180+30:180-30:2) -- cycle;
  }
}

\tikzset{
  lock/.pic={    
    \fill[even odd rule] (-2.5, 0) -- ++(0,3) -- ++(0.5,0) -- ++(0,1) arc (180:0:2) -- ++(0,-1) -- ++(0.5,0) -- ++(0,-3) -- ++(-5, 0) ++(1, 3) -- ++(0,1) arc (180:0:1.5) -- ++(0,-1) -- ++(-3,0);
  }
}  

\tikzset{
  tree/.pic={    
    \scoped[sibling distance=20, level distance=12]{
      \tikzstyle{every node}=[fill, circle, draw, inner sep=1, anchor=center];
      \tikzstyle{level 2}=[sibling distance=6];
      \fill (0,0.5) node (root) [anchor=south] {} child {node (b) {} child {node {}} child {node {}} child {node {}}} child { node {} child {node {}}};
    };  
  }
}   

% redefine comma in math mode
\AtBeginDocument{%
  \mathchardef\mathcomma\mathcode`\,
  \mathcode`\,="8000 
}
{\catcode`,=\active
  \gdef,{\mathcomma\discretionary{}{}{}}
}

