\PassOptionsToPackage{table,svgnames,dvipsnames}{xcolor}

\usepackage[utf8]{inputenc}
\usepackage[T1]{fontenc}
\usepackage[sc]{mathpazo}
\usepackage[ngerman,american]{babel}
\usepackage[autostyle]{csquotes}
\usepackage[dvipsnames]{xcolor}
\usepackage[%
  backend=biber,
  url=false,
  style=ieee,
  maxnames=4,
  minnames=3,
  maxbibnames=99,
  giveninits,
  uniquename=init]{biblatex} % TODO: adapt citation style
\usepackage{graphicx}
\usepackage{scrhack} % necessary for listings package
\usepackage{listings}
\usepackage{lstautogobble}
\usepackage{tikz}
\usepackage{pgfplots}
\usepackage{pgfplotstable}
\usepackage{booktabs}
\usepackage[final]{microtype}
\usepackage{caption}
\usepackage{subcaption}
\usepackage{amsfonts}
\usepackage{amsthm}
\usepackage{amsmath}
\usepackage{mathtools}
\usepackage[hidelinks]{hyperref} % hidelinks removes colored boxes around references and links
\usepackage[acronym]{glossaries}

\usetikzlibrary{intersections}

\def\PolynomialSSS(#1){8 + 7 * #1 - 6 * #1^2 + #1 ^ 3}%

\theoremstyle{theorem}
\newtheorem{theorem}{Theorem}[chapter]

\theoremstyle{definition}
\newtheorem{definition}{Definition}[chapter]

\bibliography{bibliography}

\setkomafont{disposition}{\normalfont\bfseries} % use serif font for headings
\linespread{1.05} % adjust line spread for mathpazo font

% Add table of contents to PDF bookmarks
\BeforeTOCHead[toc]{{\cleardoublepage\pdfbookmark[0]{\contentsname}{toc}}}

% Define TUM corporate design colors
% Taken from http://portal.mytum.de/corporatedesign/index_print/vorlagen/index_farben
\definecolor{TUMBlue}{HTML}{0065BD}
\definecolor{TUMSecondaryBlue}{HTML}{005293}
\definecolor{TUMSecondaryBlue2}{HTML}{003359}
\definecolor{TUMBlack}{HTML}{000000}
\definecolor{TUMWhite}{HTML}{FFFFFF}
\definecolor{TUMDarkGray}{HTML}{333333}
\definecolor{TUMGray}{HTML}{808080}
\definecolor{TUMLightGray}{HTML}{CCCCC6}
\definecolor{TUMAccentGray}{HTML}{DAD7CB}
\definecolor{TUMAccentOrange}{HTML}{E37222}
\definecolor{TUMAccentGreen}{HTML}{A2AD00}
\definecolor{TUMAccentLightBlue}{HTML}{98C6EA}
\definecolor{TUMAccentBlue}{HTML}{64A0C8}

% Settings for pgfplots
\pgfplotsset{compat=newest}
\pgfplotsset{
  % For available color names, see http://www.latextemplates.com/svgnames-colors
  cycle list={TUMBlue\\TUMAccentOrange\\TUMAccentGreen\\TUMSecondaryBlue2\\TUMDarkGray\\},
}


\makeglossaries

\newacronym{abe}{ABE}{Attribute-Based Encryption}
\newacronym{abes}{ABE scheme}{Attribute-Based Encryption scheme}

\newglossaryentry{gls-kgc}
{
        name={key generation center},
        text={Key Generation Center},
        description={Trusted central authority that sets up an \acrshort{abes} and generates keys for users of an \acrshort{abes} }
}

\newglossaryentry{ec}{
  name={elliptic curve},
  text={Elliptic Curve},
  description={Algebraic structure that forms a \gls{group}, see Section~\ref{sec:ec}}
}

\newglossaryentry{group}{
  name={group},
  description={A set together with a binary operation that satisfies the group axioms, see Section~\ref{sec:group}}
}

\newglossaryentry{gls-kp-abe}{
  name={key-policy \acrshort{abe}},
  description={Variant of \gls{abe} where the ciphertext is associated with an \gls{access-policy} and the key is associated with a set of \glspl{attribute}.}
}
\newacronym[see={[Glossary:]{gls-kp-abe}}]{kp-abe}{KP-ABE}{Key-Policy \acrshort{abe}\glsadd{gls-cp-abe}}
\newglossaryentry{gls-cp-abe}{
  name={ciphertext-policy \acrshort{abe}},
  description={Variant of \acrshort{abe} where the key is associated with an \gls{access-policy} and the ciphertext is associated with a set of \glspl{attribute}.}
}
\newacronym[see={[Glossary:]{gls-cp-abe}}]{cp-abe}{CP-ABE}{Ciphertext-Policy \acrshort{abe}\glsadd{gls-cp-abe}}


\newglossaryentry{attribute}{name={attribute},description={Property of an actor or object, e.g. ,,is student'' or ''has blonde hair''}}

\newglossaryentry{access-policy}{
  name={access policy},
  description={A policy that defines what combination of \glspl{attribute} shall be required to access data.}
}

\newglossaryentry{universe}{
  name={attribute universe},
  description={set of possible attributes}
}

\newglossaryentry{small-universe}{
  name={small universe},
  description={type of \acrshort{abe} construction where the possible attributes have to be fixed when the system is instantiated}
}

\newglossaryentry{large-universe}{
  name={large universe},
  description={type of \acrshort{abe} construction where any string can be used as an attribute}
}

\newglossaryentry{pkes}{
  name={public-key encryption scheme},
  description={also called asymmetric encryption scheme. Type of encryption scheme where different keys are used for encryption and decryption. The encryption key may be made public, while the decryption key is kept private.}
}

\newglossaryentry{privkes}{
  name={private-key encryption scheme},
  description={also called symmetric encryption scheme. Type of encryption scheme where the same key is used for encryption and decryption. This means that the key has to be shared among all parties via some secure channel (e.g. a personal meeting).}
}

\newacronym[see={[Glossary:]{gls-kgc}}]{kgc}{KGC}{Key Generation Center\glsadd{gls-kgc}}
% \newacronym{kgc}{KGC}{Key Generation Center}

% Settings for lstlistings
\lstset{%
  basicstyle=\ttfamily,
  columns=fullflexible,
  autogobble,
  keywordstyle=\bfseries\color{TUMBlue},
  stringstyle=\color{TUMAccentGreen}
}

\tikzset{
  key/.pic={
    \filldraw (2.7,1) -- ++(-7, 0) -- ++(-1,-1) -- ++(1, -1) -- ++(0.5, 0.5) -- ++(0.5, -0.5) -- ++(0.75, 0.75) -- ++(0.75, -0.75) -- ++(0.5, 0.5) -- ++(0.5, -0.5) -- ++(1, 1) -- ++(1, -1) -- ++(0.5, 0.5) -- ++(0.5, -0.5) -- ++(0.5, 0) arc (-180+30:180-30:2) -- cycle;
  }
}

\tikzset{
  lock/.pic={    
    \fill[even odd rule] (-2.5, 0) -- ++(0,3) -- ++(0.5,0) -- ++(0,1) arc (180:0:2) -- ++(0,-1) -- ++(0.5,0) -- ++(0,-3) -- ++(-5, 0) ++(1, 3) -- ++(0,1) arc (180:0:1.5) -- ++(0,-1) -- ++(-3,0);
  }
}                 

% redefine comma in math mode
\AtBeginDocument{%
  \mathchardef\mathcomma\mathcode`\,
  \mathcode`\,="8000 
}
{\catcode`,=\active
  \gdef,{\mathcomma\discretionary{}{}{}}
}

