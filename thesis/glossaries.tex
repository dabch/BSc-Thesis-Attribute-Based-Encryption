\newacronym{abe}{ABE}{Attribute-Based Encryption}
\newacronym{abes}{ABE scheme}{Attribute-Based Encryption scheme}

\newglossaryentry{gls-kgc}
{
        name={key generation center},
        text={Key Generation Center},
        description={Trusted central authority that sets up an \acrshort{abes} and generates keys for users of an \acrshort{abes} }
}

\newglossaryentry{ec}{
  name={elliptic curve},
  text={Elliptic Curve},
  description={Algebraic structure that forms a \gls{group}, see Section~\ref{sec:ec}}
}

\newglossaryentry{group}{
  name={group},
  description={A set together with a binary operation that satisfies the group axioms, see Section~\ref{sec:group}}
}

\newglossaryentry{gls-kp-abe}{
  name={key-policy \acrshort{abe}},
  description={Variant of \gls{abe} where the ciphertext is associated with an \gls{access-policy} and the key is associated with a set of \glspl{attribute}}
}
\newacronym[see={[Glossary:]{gls-kp-abe}}]{kp-abe}{KP-ABE}{Key-Policy \acrshort{abe}\glsadd{gls-cp-abe}}
\newglossaryentry{gls-cp-abe}{
  name={ciphertext-policy \acrshort{abe}},
  description={Variant of \acrshort{abe} where the key is associated with an \gls{access-policy} and the ciphertext is associated with a set of \glspl{attribute}}
}
\newacronym[see={[Glossary:]{gls-cp-abe}}]{cp-abe}{CP-ABE}{Ciphertext-Policy \acrshort{abe}\glsadd{gls-cp-abe}}


\newglossaryentry{attribute}{name={attribute},description={Property of an actor or object, e.g. ,,is student'' or ''has blonde hair''}}

\newglossaryentry{access-policy}{
  name={access policy},
  description={A policy that defines what combination of \glspl{attribute} shall be required to access data. Formalized by an access structure and usually realized by an access tree, see Section~\ref{sec:access-structures}}
  plural={access policies},
}

\newglossaryentry{access-structure}{
  name={access structure},
  description={defines the attribute combinations that are required and sufficient to decrypt a ciphertext. See Section~\ref{sec:access-structures} and Section~\ref{sec:access-trees}}
}

\newglossaryentry{access-tree}{
  name={access tree},
  description={construction to realize (monotone) access structures. See Section~\ref{sec:access-trees}}
}

\newglossaryentry{kdf}{name={key derivation function},description={function that derives a suitable cryptographic key from some other data, that may be too long or not in the right format to serve as a key. Usually, hash functions are used as KDF}}

\newglossaryentry{universe}{
  name={attribute universe},
  description={set of possible attributes}
}

\newglossaryentry{small-universe}{
  name={small universe},
  description={type of \acrshort{abe} construction where the possible attributes have to be fixed when the system is instantiated}
}

\newglossaryentry{large-universe}{
  name={large universe},
  description={type of \acrshort{abe} construction where any string can be used as an attribute}
}

\newglossaryentry{pkes}{
  name={asymmetric encryption scheme},
  description={type of encryption scheme where different keys are used for encryption and decryption. The encryption key may be made public, while the decryption key is kept private.}
}

\newglossaryentry{privkes}{
  name={symmetric encryption scheme},
  description={type of encryption scheme where the same key is used for encryption and decryption. This means that the key has to be shared among all parties via some secure channel (e.g. a personal meeting).}
}

\newacronym[see={[Glossary:]{gls-kgc}}]{kgc}{KGC}{Key Generation Center\glsadd{gls-kgc}}
% \newacronym{kgc}{KGC}{Key Generation Center}

\newglossaryentry{ibe}{
  name={identity-based encryption},
  description={type of encryption where data is encrypted using a unique identity (e.g. an email address or phone number), and only the participant holding the secret key corresponding to that identity is able to decrypt the ciphertext.}
}

\newglossaryentry{gls-msp}{
  name={monotone span program},
  description={linear algebraic computation model that is equivalent to \acrshort{lsss}. See Section~\ref{sec:lsss}},
}

\newglossaryentry{gls-lsss}{
  name={linear secret sharing scheme},
  description={secret sharing scheme in which the share generation can be described by a matrix. See~\cite{beimel_secure_1996}}
}

\newacronym[see={[Glossary:]{gls-msp}}]{msp}{MSP}{monotone span program\glsadd{gls-msp}}
\newacronym[see={[Glossary:]{gls-lsss}}]{lsss}{LSSS}{linear secret sharing scheme\glsadd{gls-lsss}}

\newglossaryentry{standard-model}{
  name={standard model},
  description={formal security model that imposes no restrictions on the attacker, except for the limit on the complexity of their computations}
}

\newglossaryentry{ggm}{
  name={generic group model},
  description={formal security model assuming that the attacker only has oracle access to the group operation (provides less formal security than the \gls{standard-model})}
}

\newglossaryentry{dss}{name={digital signature scheme},description={asymmetric cryptographic scheme/protocol for ensuring message authenticity and integrity}}

\newglossaryentry{dh}{
  name={diffie-hellman key exchange},
  text={Diffie-Hellman Key Exchange},
  description={key agreement protocol that allows two parties A and B to agree on a shared secret value over an insecure channel. A and B can derive the same secret value, while any adversaries cannot (as long as they only passively eavesdrop, but not modify, the information exchanged between A and B)}
}

\newglossaryentry{security-level}{
  name={security level},
  description={ % TODO check this and find a citation
    measure of the strength of a cryptographic scheme, usually given in bits.
    A security level of $n$ bits means that the most efficient attack needs to perform at least around $2^n$ operations to break the scheme.
    Note that this does not directly translate to the size of the used parameters: 
    to guarantee a security level of $n$ bits, usually the the size of the field underlying our elliptic curve (i.e. the number of bits of its modulus) needs to be \emph{at least} $2n$, sometimes much larger},
}

\newacronym[description={single instruction, multiple data. Vectorized CPU instruction that processes several pieces of data at once}]{simd}{SIMD}{single instruction multiple data}

\newacronym{iot}{IoT}{Internet of Things}